\documentclass[12pt,a4paper]{article}
 % document definitions
% packages
\usepackage[brazilian]{babel}
\usepackage[utf8]{inputenc}
\usepackage[T1]{fontenc}
\usepackage[usenames]{color}
\usepackage{lscape}
\usepackage{latexsym}
\usepackage{amscd}
\usepackage{amsfonts}
\usepackage{epsf}
\usepackage{times}
\usepackage{multicol}
\usepackage[pdftex]{graphicx}
%\usepackage[makeidx]
\usepackage{multirow}
%\usepackage{thumbpdf}
\usepackage{float}
\usepackage{paralist}
\usepackage{alltt}
\usepackage{mdwlist}
%\usepackage{color}
%\usepackage{rotating}
\usepackage{url}
\usepackage{hyperref}
\usepackage{amssymb}

% toc
\setcounter{tocdepth}{2}
\setlength{\textwidth}{40pc}
\setlength{\textheight}{56pc}
\setlength{\parskip}{.3pc}
\setlength{\topmargin}{-17mm}
\setlength{\oddsidemargin}{1mm}
\setlength{\evensidemargin}{1mm}
\linespread{1.5}

% float settings
\floatstyle{boxed}
\newfloat{program}{htp}{lop}
\floatname{program}{Programa}

% page formatting
\pagestyle{plain}

% rename the chapter name
% http://www.tex.ac.uk/cgi-bin/texfaq2html?label=fixnam
\addto\captionsbrazilian{%
  \renewcommand{\chaptername}%
    {Seção}%
}

\begin{document}
% include cover

\begin{titlepage}

% command aliases
\newcommand{\HRule}{\rule{\linewidth}{0.5mm}} 
 
\begin{center}
 
% Upper part of the page
%\includegraphics[width=0.15\textwidth]{./logo}\\[1cm]

%\HRule \\
\textsc{\LARGE Universidade Estadual de Campinas}\\[0.5cm]
\textsc{\Large Laboratório de Redes - MC823}
%\HRule \\
{\ }\\[4.5cm]
 
% Title
%{\ }\\[2.0cm]
{ \huge \bfseries Tarefa 2}\\[3.0cm]
 
% Student and advisor
\begin{flushright}
Miguel Francisco Alves de Mattos Gaiowski \\
\emph{RA 076116} \\
Guillaume Massé\\
\emph{RA 107888} \\[2.0cm]


\end{flushright}


\begin{flushright}
Prof. Paulo Lício de Geus \\
\end{flushright}
 
\vfill
 
% Bottom of the page
{\large Campinas, \today}
 
\end{center}
 
\end{titlepage}

\tableofcontents

%%%%%%%%%%%%%%%%%%%%%%%%%%%%
\section{Introdução}
%%%%%%%%%%%%%%%%%%%%%%%%%%%%

Usando os programas do experimento anterior, o enunciado pedia para a
primeira fase que os comandos de \textit{send} e \textit{recv} fossem
trocados por \textit{fputs} e \textit{fgets}, respectivamente. Para a
segunda fase, deveríamos notar que a demora no envio dos arquivos era
devida ao RTT e modificar o cliente para que fosse dividido em dois
processos e não ficasse esperando resposta do servidor para continuar
enviando dados.

%%%%%%%%%%%%%%%%%%%%%%%%%%%%
\section{Objetivos}
%%%%%%%%%%%%%%%%%%%%%%%%%%%%

Utilizar descritores de arquivos para envio e recebimento de dados e
separar o cliente em dois processos (pai e filho), dividindo também o
trabalho de enviar e recber dados, para maximizar o uso de banda.

%%%%%%%%%%%%%%%%%%%%%%%%%%%%
\section{Desenvolvimento}
%%%%%%%%%%%%%%%%%%%%%%%%%%%%

A primeira fase do projeto foi de fácil implementação. Primeiro
precisamos usar a função \textit{fdopen} para conseguir descritores de
arquivos de leitura e escrita ligados ao socket conectado. Com isso
pudemos substituir as funções \textit{send} e \textit{recv} por
\textit{fputs} e \textit{fputs}. Obviamente alguns ajustes tiveram que
ser feitos. Um exemplo disso é a detecção que o cliente desconectou-se
por parte do servidor. Tivemos que usar a saída da função
\textit{fgets}, que quando é igual a \textit{NULL} significa que o
cliente fechou a conexão.

A segunda parte foi mais trabalhosa. Precisamos primeiramente entender
como que poderíamos separar o cliente em dois. Como o enunciado
sugeria, fizemos com que o processo filho ficasse encarregado de
enviar as linhas da entrada padrão para o descritor de escrita e que o
processo pai fosse responsável por receber as linhas retornadas pelo
servidor pelo descritor de leitura.

Obviamente o processo pai deve esperar o processo filho terminar para
que imprima suas estatísticas e retorne, já que ele depende do
processo filho ter enviado tudo e o servidor enviado de volta. 

%%%%%%%%%%%%%%%%%%%%%%%%%%%%
\section{Dificuldades}
%%%%%%%%%%%%%%%%%%%%%%%%%%%%

Já que o processo filho não compartilha variáveis com o processo pai,
e somente os descritores abertos, não temos como fazer com que o pai
saiba quantas linhas o filho enviou para o servidor. Assim, tivemos
que fazer com que o filho imprima suas estatísticas de envio e o
processo pai imprima as de recebimento e o tempo total. Isto nos tomou
um certo tempo para perceber que seria a solução mais simples.

Deve-se fazer um half-close na conexão pelo filho. Ou seja, o
descritor de escrita é fechado, mas o de leitura continua aberto. Se o
filho fechar ambos os descritores o pai não consegue receber as
mensagens do servidor, já que a conexão foi fechada.

Além disso, o filho deve terminar com \textit{\_exit(0)} e não
\textit{exit(0)}, pois esta última fecha todos descritores e sockets
abertos no processo, o que não queremos.

%%%%%%%%%%%%%%%%%%%%%%%%%%%%
\section{Experimentos}
%%%%%%%%%%%%%%%%%%%%%%%%%%%%

Para testar os programas, usamos uma máquina no Canadá (Montreal) e outra no IC-3
(nossos notebooks). Como as redes são protegidas por firewall, tivemos
que usar túneis SSH para ligar o cliente ao servidor. 

O tempo de RTT entre as máquinas que colocamos os processos servidor e
cliente foi de 175 ms, medidos com ping, como pode ser conferido abaixo.

{\scriptsize 
\begin{verbatim}
[gumasa@l4712-19 ~]$ ifconfig eth0 | grep addr
eth0      Link encap:Ethernet  HWaddr 00:30:48:7E:B2:2C  
          inet addr:132.207.12.51  Bcast:132.207.12.255  Mask:255.255.255.0
          inet6 addr: fe80::230:48ff:fe7e:b22c/64 Scope:Link
[gumasa@l4712-19 ~]$ ping bugu.lab.ic.unicamp.br
PING bugu.lab.ic.unicamp.br (143.106.16.165) 56(84) bytes of data.
64 bytes from bugu.lab.ic.unicamp.br (143.106.16.165): icmp_seq=1 ttl=45 time=175 ms
64 bytes from bugu.lab.ic.unicamp.br (143.106.16.165): icmp_seq=2 ttl=45 time=175 ms
64 bytes from bugu.lab.ic.unicamp.br (143.106.16.165): icmp_seq=3 ttl=45 time=174 ms
64 bytes from bugu.lab.ic.unicamp.br (143.106.16.165): icmp_seq=4 ttl=45 time=175 ms
64 bytes from bugu.lab.ic.unicamp.br (143.106.16.165): icmp_seq=5 ttl=45 time=175 ms
^C
--- bugu.lab.ic.unicamp.br ping statistics ---
5 packets transmitted, 5 received, 0% packet loss, time 8855ms
rtt min/avg/max/mdev = 174.872/175.333/175.850/0.616 ms
\end{verbatim}}

Sabendo que nosso cliente da fase 1 manda uma linha por vez e espera a
resposta antes de enviar a próxima, podemos fazer uma previsão de
tempo para mandar o arquivo \textit{/etc/services}. Usando \textit{wc}
nós conseguimos a informação sobre a quantidade de linhas do arquivo.

\begin{verbatim}
$ wc -l /etc/services 
   13921 /etc/services
\end{verbatim}

Como cada linha deve levar pelo menos um RTT para ser enviada e
recebida pelo cliente, podemos fazer uma previsão do tempo mínimo usando a fórmula $NL
* RTT $, onde $NL$ é o número de linhas e $RTT$ é o round trip time.

$13921 * 175 = 2436175 ms = 2436 s = 40.5 min$

Executando o teste da fase 1, do lado do cliente:

\begin{verbatim}
$ ./client localhost < /etc/services > s1
Received: Conectado, envie uma mensagem que eu devolvo.
Tempo total de transferencia: 2501.847383 s
Numero de linhas enviadas: 13921
Numero de linhas recebidas: 13921
Numero de caracteres na maior linha: 118
Total de caracteres enviados: 677959
Total de caracteres recebidos: 677959
caesar:fase1 miguelgaiowski$ diff /etc/services s1
caesar:fase1 miguelgaiowski$ wc /etc/services 
   13921   68850  677959 /etc/services
\end{verbatim}

E do lado do servidor: 

\begin{verbatim}
[gumasa@l4712-19 fase1]$ ./server_echo 
Servidor: Conectado com 127.0.0.1
Cliente fechou a conexão.
  Numero de linhas recebidas: 13921
  Total de caracteres recebidos: 677959
\end{verbatim}

O tempo obtido foi de 2500 segundos. Um erro de apenas 2.5\% da nossa
estimativa. É claro que já esperávamos um tempo um pouco maior, já que
estamos usando um túnel SSH e cada mensagem deve ser cifrada. 

Já na fase 2 o cliente não fica esperando resposta para enviar a
próxima linha para o servidor. Assim, o tempo não depende mais do RTT,
e sim da banda. Como vamos usar todo o potencial da banda, sem ficar
esperando respostas, esperamos que seja extremamente mais rápido para
enviar e receber o arquivo. Executando o teste, primeiro do lado do cliente:

\begin{verbatim}
Gui:fase2 Gui$ ./client 127.0.0.1 < /etc/services > t2
Received: Conectado, envie uma mensagem que eu devolvo.
Numero de linhas enviadas: 13921
Numero de caracteres na maior linha: 118
Total de caracteres enviados: 677959
Tempo total de transferencia: 4.608163 s
Numero de linhas recebidas: 13921
Total de caracteres recebidos: 677959
Gui:fase2 Gui$ diff /etc/services t2
Gui:fase2 Gui$ wc /etc/services 
   13921   68850  677959 /etc/services
\end{verbatim}

E do lado do servidor:

\begin{verbatim}
[gumasa@l3818-19 fase2]$ ./server_echo 
Servidor: Conectado com 127.0.0.1
Cliente fechou a conexão.
  Numero de linhas recebidas: 13921
  Total de caracteres recebidos: 677959
\end{verbatim}

Nota-se que o ganho de performance é incrível. Mais de 500 vezes mais
rápido.

Testamos também os programas da experimento anterior, para fazer um
comparativo.

\begin{verbatim}
caesar:lab1 miguelgaiowski$ ./client localhost < /etc/services > t1
Received: Conectado, envie uma mensagem que eu devolvo.
Tempo total de transferencia: 2507.296164 s
Numero de linhas enviadas: 13921
Numero de linhas recebidas: 13921
Numero de caracteres na maior linha: 118
Total de caracteres enviados: 677959
Total de caracteres recebidos: 677959
\end{verbatim}

Com isso, notamos que usar \textit{fgets} e \textit{fputs} é pouca
coisa mais eficiente que \textit{send} e \textit{recv}.

Segue uma tabela comparativa:

\begin{tabular}{|l|r|r|r|}
\hline
Programa & Lab 1 & Lab 2 - Fase 1 & Lab 2 - Fase 2  \\ \hline
Tempo (s) & 2507 & 2501 & 4.6 \\ \hline
\end{tabular}

%%%%%%%%%%%%%%%%%%%%%%%%%%%%
\section{Conclusões}
%%%%%%%%%%%%%%%%%%%%%%%%%%%%

Com os testes efetuados pudemos notar claramente a dependência no RTT
do cliente simples que enviava uma linha somente depois de ter a
resposta da anterior. 

Notamos que uma melhoria de design tão pequena quanto essa pode trazer
uma melhora de performance extremamente notável. No nosso caso, da
ordem de 500 vezes mais rápido.

%%%%%%%%%%%%%%%%%%%%%%%%%%%%
\section{Bibliografia}
%%%%%%%%%%%%%%%%%%%%%%%%%%%%
\nocite{beej}

\bibliographystyle{plain}
\bibliography{bibliography}

%%%%%%%%%%%%%%%%%%%%%%%%%%%%
\section{Anexos}
%%%%%%%%%%%%%%%%%%%%%%%%%%%%

Os códigos dos programas seguem anexos.

\end{document}


