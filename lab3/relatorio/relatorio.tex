\documentclass[12pt,a4paper]{article}
 % document definitions
% packages
\usepackage[brazilian]{babel}
\usepackage[utf8]{inputenc}
\usepackage[T1]{fontenc}
\usepackage[usenames]{color}
\usepackage{lscape}
\usepackage{latexsym}
\usepackage{amscd}
\usepackage{amsfonts}
\usepackage{epsf}
\usepackage{times}
\usepackage{multicol}
\usepackage[pdftex]{graphicx}
%\usepackage[makeidx]
\usepackage{multirow}
%\usepackage{thumbpdf}
\usepackage{float}
\usepackage{paralist}
\usepackage{alltt}
\usepackage{mdwlist}
%\usepackage{color}
%\usepackage{rotating}
\usepackage{url}
\usepackage{hyperref}
\usepackage{amssymb}

% toc
\setcounter{tocdepth}{2}
\setlength{\textwidth}{40pc}
\setlength{\textheight}{56pc}
\setlength{\parskip}{.3pc}
\setlength{\topmargin}{-17mm}
\setlength{\oddsidemargin}{1mm}
\setlength{\evensidemargin}{1mm}
\linespread{1.5}

% float settings
\floatstyle{boxed}
\newfloat{program}{htp}{lop}
\floatname{program}{Programa}

% page formatting
\pagestyle{plain}

% rename the chapter name
% http://www.tex.ac.uk/cgi-bin/texfaq2html?label=fixnam
\addto\captionsbrazilian{%
  \renewcommand{\chaptername}%
    {Seção}%
}

\begin{document}
% include cover

\begin{titlepage}

% command aliases
\newcommand{\HRule}{\rule{\linewidth}{0.5mm}} 
 
\begin{center}
 
% Upper part of the page
%\includegraphics[width=0.15\textwidth]{./logo}\\[1cm]

%\HRule \\
\textsc{\LARGE Universidade Estadual de Campinas}\\[0.5cm]
\textsc{\Large Laboratório de Redes - MC823}
%\HRule \\
{\ }\\[4.5cm]
 
% Title
%{\ }\\[2.0cm]
{ \huge \bfseries Tarefa 2}\\[3.0cm]
 
% Student and advisor
\begin{flushright}
Miguel Francisco Alves de Mattos Gaiowski \\
\emph{RA 076116} \\
Guillaume Massé\\
\emph{RA 107888} \\[2.0cm]


\end{flushright}


\begin{flushright}
Prof. Paulo Lício de Geus \\
\end{flushright}
 
\vfill
 
% Bottom of the page
{\large Campinas, \today}
 
\end{center}
 
\end{titlepage}

\tableofcontents

%%%%%%%%%%%%%%%%%%%%%%%%%%%%
\section{Introdução}
%%%%%%%%%%%%%%%%%%%%%%%%%%%%

Enquanto as outras tarefas tratavam de conexões TCP, esta trata de UDP.
Faremos um servidor de eco, e um cliente para interagir com ele. 
Além disso, usaremos diferentes métodos para utilizar UDP com a linguagem C.
Isto é, usaremos as funções send(), recv(), sendto(), recvfrom() e connect().

%%%%%%%%%%%%%%%%%%%%%%%%%%%%
\section{Objetivos}
%%%%%%%%%%%%%%%%%%%%%%%%%%%%

Implementar um cliente e um servidor de eco usando datagramas. Verificar a falta de confiabilidade de UDP.
Testar as diferentes combinações de métodos para a implementação.

%%%%%%%%%%%%%%%%%%%%%%%%%%%%
\section{Desenvolvimento}
%%%%%%%%%%%%%%%%%%%%%%%%%%%%

Implementamos um cliente em UDP. Basicamente ele lê linhas da entrada padrão e as transmite para o servidor. Após cada linha ele espera um resposta do servidor para enviar a próxima. 

O servidor recebe as linhas e as envia para o cliente. 

Vale notar que fizemos duas implementações diferentes para o cliente e para o servidor. Uma delas usa o par sendto e recvfrom. A outra usa send e recv após um connect. Sabemos que o termo connect não se aplica a UDP, sendo uma nomenclatura infeliz da biblioteca de C. Ainda assim, ela simula uma conexão após o primeiro datagrama ser recebido e permite que se possa continuar recebendo sem falar de quem, basta fazer uma conexão virtual antes.

No servidor, caso esteja recebendo dados de um cliente e passe mais de um segundo sem receber nada, um sinal de alarme é emitido e o servidor trata imprimindo as estatísticas obtidas até o momento e reiniciando a contagem. Isso simula que um novo cliente será o próximo a enviar dados. Isso pode acontecer quando um cliente deixa de enviar um pacote final com 0 bytes indicando fim de troca de dados entre eles. 

%%%%%%%%%%%%%%%%%%%%%%%%%%%%
\section{Dificuldades}
%%%%%%%%%%%%%%%%%%%%%%%%%%%%

Tivemos algumas dificuldades neste laboratório para fazer o tratamento de sinais de alarme, já que nunca tínhamos feito algo parecido. Google foi nosso parceiro retornando exemplos de uso.

Tivemos problemas também usando um cliente com sendto/recvfrom e um servidor com connect ao mesmo tempo. 

%%%%%%%%%%%%%%%%%%%%%%%%%%%%
\section{Experimentos}
%%%%%%%%%%%%%%%%%%%%%%%%%%%%

Testamos os programas em máquinas do IC-3. O tempo de transmissão do arquivo /etc/services variou conforme os diferente métodos usados. Já que o tempo foi muito pequeno, geramos um arquivo 10 vezes maior, fazendo concatenando o /etc/services nele mesmo. O arquivo que geramos tinha 106345 linhas. 

Usando as funções sendto()/recvfrom() tanto no cliente quanto no servidor, obtivemos uma medição de 59.356 segundos para envio do arquivo /etc/services entre duas máquinas do IC-3.

Com o cliente usando connect() e depois send()/recv(), o tempo foi de 58.812 segundos. Já com os papéis trocados, ou seja, servidor fazendo connect(), observamos um tempo de 59.311 segundos. Uma mudança praticamente desprezavel.

Fizemos um flood.c para testar a perda de pacotes. Deixamos vários processos fazendo cálculos inúteis na mesma máquina do servidor. No cliente, retiramos a espera pela resposta antes de envio da próxima linha. Além disso, omitimos todo tipo de output, para que enviasse todas as linhas o mais rápido possível.

O servidor não conseguiu receber todos os pacotes tão rápido quanto o cliente enviava e, já que não há reenvio, acabou perdendo vários pacotes. Ele recebeu apenas 77605 linhas, das 106345. Ou seja, apenas 72,9\% dos pacotes.

%%%%%%%%%%%%%%%%%%%%%%%%%%%%
\section{Conclusões}
%%%%%%%%%%%%%%%%%%%%%%%%%%%%

Notamos que UDP não é confiável e é tão simples de ser usado
quanto TCP. Podemos concluir, no entanto, que UDP pode ser muito
bom para certas aplicações, tais como streaming de áudio e vídeo,
graças a sua versatilidade e habilidade de manter uma velocidade
constante de envio e outra de recebimento, bastando tratar a perda
de pacotes. 

%%%%%%%%%%%%%%%%%%%%%%%%%%%%
\section{Bibliografia}
%%%%%%%%%%%%%%%%%%%%%%%%%%%%
\nocite{beej}

\bibliographystyle{plain}
\bibliography{bibliography}

%%%%%%%%%%%%%%%%%%%%%%%%%%%%
\section{Anexos}
%%%%%%%%%%%%%%%%%%%%%%%%%%%%

Os códigos dos programas seguem anexos.

\end{document}


