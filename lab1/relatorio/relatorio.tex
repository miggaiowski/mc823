\documentclass[12pt,a4paper]{article}
 % document definitions
% packages
\usepackage[brazilian]{babel}
\usepackage[utf8]{inputenc}
\usepackage[T1]{fontenc}
\usepackage[usenames]{color}
\usepackage{lscape}
\usepackage{latexsym}
\usepackage{amscd}
\usepackage{amsfonts}
\usepackage{epsf}
\usepackage{times}
\usepackage{multicol}
\usepackage[pdftex]{graphicx}
%\usepackage[makeidx]
\usepackage{multirow}
%\usepackage{thumbpdf}
\usepackage{float}
\usepackage{paralist}
\usepackage{alltt}
\usepackage{mdwlist}
%\usepackage{color}
%\usepackage{rotating}
\usepackage{url}
\usepackage{hyperref}
\usepackage{amssymb}

% toc
\setcounter{tocdepth}{2}
\setlength{\textwidth}{40pc}
\setlength{\textheight}{56pc}
\setlength{\parskip}{.3pc}
\setlength{\topmargin}{-17mm}
\setlength{\oddsidemargin}{1mm}
\setlength{\evensidemargin}{1mm}
\linespread{1.5}

% float settings
\floatstyle{boxed}
\newfloat{program}{htp}{lop}
\floatname{program}{Programa}

% page formatting
\pagestyle{plain}

% rename the chapter name
% http://www.tex.ac.uk/cgi-bin/texfaq2html?label=fixnam
\addto\captionsbrazilian{%
  \renewcommand{\chaptername}%
    {Seção}%
}

\begin{document}
% include cover

\begin{titlepage}

% command aliases
\newcommand{\HRule}{\rule{\linewidth}{0.5mm}} 
 
\begin{center}
 
% Upper part of the page
%\includegraphics[width=0.15\textwidth]{./logo}\\[1cm]

%\HRule \\
\textsc{\LARGE Universidade Estadual de Campinas}\\[0.5cm]
\textsc{\Large Laboratório de Redes - MC823}
%\HRule \\
{\ }\\[4.5cm]
 
% Title
%{\ }\\[2.0cm]
{ \huge \bfseries Tarefa 1}\\[3.0cm]
 
% Student and advisor
\begin{flushright}
Miguel Francisco Alves de Mattos Gaiowski \\
\emph{RA 076116} \\
Guillaume Massé\\
\emph{RA 107888} \\[2.0cm]


\end{flushright}


\begin{flushright}
Prof. Paulo Lício de Geus \\
\end{flushright}
 
\vfill
 
% Bottom of the page
{\large Campinas, \today}
 
\end{center}
 
\end{titlepage}

\tableofcontents

%%%%%%%%%%%%%%%%%%%%%%%%%%%%
\section{Introdução}
%%%%%%%%%%%%%%%%%%%%%%%%%%%%

Esta primeira tarefa pedia que um servidor de eco fosse implementado
usando a linguagem C. Além disso, um cliente também deveria ser feito
com o intuito de se conectar ao servidor e enviar mensagens e receber
o que o servidor manda de volta.

Além de um servidor simples, que só aceita uma conexão por vez,
deveríamos também implementar um que funcionasse de forma concorrente,
aceitando várias conexões ao mesmo tempo e fazendo o tratamento adequado.

%%%%%%%%%%%%%%%%%%%%%%%%%%%%
\section{Objetivos}
%%%%%%%%%%%%%%%%%%%%%%%%%%%%

Depois dos servidores e do cliente prontos, deveríamos poder fazer
testes simples, como enviar e receber de volta de a mesma mensagem,
como também fazer o \textit{pipe} de arquivos para a entrada e saída
do cliente e analisar o tempo levado para a transmissão dos dados. Com
isso também conseguiríamos averiguar o funcionamento correto dos dois
lados da conexão, se o cliente realmente recebe de volta exatamente o
que mandou.

%%%%%%%%%%%%%%%%%%%%%%%%%%%%
\section{Desenvolvimento}
%%%%%%%%%%%%%%%%%%%%%%%%%%%%

Protótipos do cliente e servidor foram disponibilizados para que
modificassemos. A parte do cliente foi relativamente fácil, bastando
pegar a conexão fazer a entrada e saída de dados. Os dados da entrada
padrão foram jogados no socket para o servidor, e após fechar a
conexão alguns dados estatísticos da 'conversa' entre o cliente e o
servidor são impressos na saída de erro. 

Para o servidor de eco simples, que só aceita uma conexão por vez,
tivemos que pegar o protótipo e fazer com que a conexão não fosse
fechada após receber a primeira mensagem, como estava implementado. Ao
invés disso a conexão fica aberta até que o cliente a feche (número de
bytes recebidos seja zero). Além disso, após o fechamento da conexão o
servidor deve entrar a em loop e aguardar novos pedidos de conexão. 

Após receber cada mensagem, o servidor envia exatamente a mesma coisa
para o cliente. E quando o cliente fecha a conexão, o servidor joga na
saída de erro algumas informações sobre a quantidade de dados enviada
e recebida.

Com o servidor concorrente, após entender como funciona o processor de
fork para que o processo filho cuide de cada conexão recebida enquanto
o processo pai fica aguardando novas conexões, o trabalho foi
basicamente o mesmo, de implementar as funcionalidades básicas
requeridas, de fazer eco dos dados recebidos e imprimir algumas
informações ao final. O protótipo usado foi o fornecido pelo Beej \cite{beej}.

Primeiros testes foram feitos na mesma máquina, com o servidor ouvindo
numa porta e o cliente se conectando nessa porta em localhost. Algumas
mensagens foram enviadas e recebidas, e depois o teste foi feito
usando arquivos para entrada e saída, permitindo comparar com diff e
fazer contagem de caracteres com wc.

%%%%%%%%%%%%%%%%%%%%%%%%%%%%
\section{Dificuldades}
%%%%%%%%%%%%%%%%%%%%%%%%%%%%

Como o enunciado pedia que testes fossem feitos em sub-redes
distintas, colocamos o servidor rodando na máquina bugu, do IC-3, e o
cliente rodando numa máquina conectada a rede do LAS. 
A porta 3401, que estava designada nos protótipos não pode ser
acessada de uma rede fora da do IC-3, e por isso acabamos fazendo um
túnel SSH e jogando os dados no túnel, que chegavam até o servidor e
voltavam. Isso acabou fazendo com que os dados levassem mais tempo
para ir e voltar, devido ao overhead de cifragem e de encapsular os pacotes.

%%%%%%%%%%%%%%%%%%%%%%%%%%%%
\section{Conclusões}
%%%%%%%%%%%%%%%%%%%%%%%%%%%%

Pudemos aprender sobre o funcionamento de sockets e de conexões
concorrentes. O mais interessante foi ver quão simples um servidor de
eco concorrente pode ser, e que mesmo assim problemas básicos
acontecem e um entendimento da estrutura da rede, com suas camadas e
estruturas de programação, deve ser obtido para que o trabalho possa
ser concluído.

O resultado do teste de mandar o arquivo services para o servidor pode
ser visto abaixo:

\begin{verbatim}
$ ./client localhost < /etc/services > teste
Received: Conectado, envie uma mensagem que eu devolvo.
Tempo total de transferencia: 6.858977 s
Numero de linhas enviadas: 13921
Numero de linhas recebidas: 13921
Numero de caracteres na maior linha: 118
Total de caracteres enviados: 677959
Total de caracteres recebidos: 677959
$ wc /etc/services 
13921   68850  677959 /etc/services
$
\end{verbatim}

E o que o lado do servidor mostra:

\begin{verbatim}
[ra076116@bugu lab1]$ ./server_echo 
Servidor: Conectado com 127.0.0.1
Cliente fechou a conexão.
  Numero de linhas recebidas: 13921
  Total de caracteres recebidos: 677959
\end{verbatim}

Mostramos assim que o servidor de eco simples, para só um cliente por
vez funciona como esperado. 

Para mostrar o funcionamento do servidor concorrente, fizemos dois
túneis. Ambos direcionados a porta 9000 do servidor, mas um saindo da
porta 9000 e outro da porta 9001 da máquina onde estava o cliente.

Um dos clientes mostrou a saída:

\begin{verbatim}
$ ./client localhost < /etc/services > teste
Received: Conectado, envie uma mensagem que eu devolvo.
Tempo total de transferencia: 7.096509 s
Numero de linhas enviadas: 13921
Numero de linhas recebidas: 13921
Numero de caracteres na maior linha: 118
Total de caracteres enviados: 677959
Total de caracteres recebidos: 677959
caesar:lab1 miguelgaiowski$ diff teste /etc/services 
$ wc /etc/services 
13921   68850  677959 /etc/services
$
\end{verbatim}

E o outro: 

\begin{verbatim}
$ ./client localhost < /etc/services > teste2
Received: Conectado, envie uma mensagem que eu devolvo.
Tempo total de transferencia: 6.858977 s
Numero de linhas enviadas: 13921
Numero de linhas recebidas: 13921
Numero de caracteres na maior linha: 118
Total de caracteres enviados: 677959
Total de caracteres recebidos: 677959
caesar:lab1 miguelgaiowski$ diff teste2 /etc/services 
$ wc /etc/services 
13921   68850  677959 /etc/services
$
\end{verbatim}

O servidor reportou:

\begin{verbatim}
[ra076116@bugu lab1]$ ./serverb 
Servidor: Conectado com 127.0.0.1
Servidor: Conectado com 127.0.0.1
Cliente fechou a conexão.
 Numero de linhas recebidas: 13921
 Total de caracteres recebidos: 677959
Cliente fechou a conexão.
 Numero de linhas recebidas: 13921
 Total de caracteres recebidos: 677959
\end{verbatim}

Podemos notar que o envio do mesmo arquivo para o servidor simples
demorou mais do que mandar ao mesmo tempo duas vezes para o servidor
concorrente. Concluímos que algum tráfego na rede inesperado possa ter
causado a demora adicional.

%%%%%%%%%%%%%%%%%%%%%%%%%%%%
\section{Bibliografia}
%%%%%%%%%%%%%%%%%%%%%%%%%%%%
\bibliographystyle{plain}
\bibliography{bibliography}

%%%%%%%%%%%%%%%%%%%%%%%%%%%%
\section{Anexos}
%%%%%%%%%%%%%%%%%%%%%%%%%%%%

Os códigos dos programas seguem anexos.

\end{document}


