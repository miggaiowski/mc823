\documentclass[12pt,a4paper]{article}
 % document definitions
% packages
\usepackage[brazilian]{babel}
\usepackage[utf8]{inputenc}
\usepackage[T1]{fontenc}
\usepackage[usenames]{color}
\usepackage{lscape}
\usepackage{latexsym}
\usepackage{amscd}
\usepackage{amsfonts}
\usepackage{epsf}
\usepackage{times}
\usepackage{multicol}
\usepackage[pdftex]{graphicx}
%\usepackage[makeidx]
\usepackage{multirow}
%\usepackage{thumbpdf}
\usepackage{float}
\usepackage{paralist}
\usepackage{alltt}
\usepackage{mdwlist}
%\usepackage{color}
%\usepackage{rotating}
\usepackage{url}
\usepackage{hyperref}
\usepackage{amssymb}

% toc
\setcounter{tocdepth}{2}
\setlength{\textwidth}{40pc}
\setlength{\textheight}{56pc}
\setlength{\parskip}{.3pc}
\setlength{\topmargin}{-17mm}
\setlength{\oddsidemargin}{1mm}
\setlength{\evensidemargin}{1mm}
\linespread{1.5}

% float settings
\floatstyle{boxed}
\newfloat{program}{htp}{lop}
\floatname{program}{Programa}

% page formatting
\pagestyle{plain}

% rename the chapter name
% http://www.tex.ac.uk/cgi-bin/texfaq2html?label=fixnam
\addto\captionsbrazilian{%
  \renewcommand{\chaptername}%
    {Seção}%
}

\begin{document}
% include cover

\begin{titlepage}

% command aliases
\newcommand{\HRule}{\rule{\linewidth}{0.5mm}} 
 
\begin{center}
 
% Upper part of the page
%\includegraphics[width=0.15\textwidth]{./logo}\\[1cm]

%\HRule \\
\textsc{\LARGE Universidade Estadual de Campinas}\\[0.5cm]
\textsc{\Large Laboratório de Redes - MC823}
%\HRule \\
{\ }\\[4.5cm]
 
% Title
%{\ }\\[2.0cm]
{ \huge \bfseries Tarefa 1}\\[3.0cm]
 
% Student and advisor
\begin{flushright}
Miguel Francisco Alves de Mattos Gaiowski \\
\emph{RA 076116} \\
Guillaume Massé\\
\emph{RA 107888} \\[2.0cm]


\end{flushright}


\begin{flushright}
Prof. Paulo Lício de Geus \\
\end{flushright}
 
\vfill
 
% Bottom of the page
{\large Campinas, \today}
 
\end{center}
 
\end{titlepage}

\tableofcontents

%%%%%%%%%%%%%%%%%%%%%%%%%%%%
\section{Introdução}
%%%%%%%%%%%%%%%%%%%%%%%%%%%%

Esta tarefa pede que o cliente da tarefa 2 seja modificado para não usar mais fork() e sim select(). 

Além disso, o servidor deve ser transformado em um daemon.

%%%%%%%%%%%%%%%%%%%%%%%%%%%%
\section{Objetivos}
%%%%%%%%%%%%%%%%%%%%%%%%%%%%

Verificar que um cliente multiplexado usando select() é tão bom quanto um que use vários processos para cuidar de cada tarefa.

Transformar o servidor em um daemon.

%%%%%%%%%%%%%%%%%%%%%%%%%%%%
\section{Desenvolvimento}
%%%%%%%%%%%%%%%%%%%%%%%%%%%%

Usando o tutorial do Beej \cite{beej}, aprendemos sobre o funcionamento do método select(). Fizemos as modificações necessárias para que o cliente deixasse de usar fork() e fosse só um processo que multiplexava os canais de comunicação.

No servidor, usando a documentação encontrada na internet, conseguimos fazer com que o servidor fosse um daemon. Ou seja, continuasse rodando mesmo que o usuário fizesse logout. Para isso vários passos devem ser seguidos. Entre eles, desacoplar de terminais, escrever em log, fazer fork e matar o processo pai, etc. 

%%%%%%%%%%%%%%%%%%%%%%%%%%%%
\section{Dificuldades}
%%%%%%%%%%%%%%%%%%%%%%%%%%%%

A principal dificuldade desta tarefa foi aprender o funcionamento do método select(). Graças ao tutorial do Beej \cite{beej} conseguimos entender o funcionamento e implementar o que foi pedido.

Além disso, outro problema que acontece é quanto a bufferização. É necessário que usemos a função setvbuf() para modificar o modo de buffer.

%%%%%%%%%%%%%%%%%%%%%%%%%%%%
\section{Experimentos}
%%%%%%%%%%%%%%%%%%%%%%%%%%%%

Os programas foram testados com uma máquina em casa e outra do IC-3. 
O RTT medido entre as máquinas foi de 150 milisegundos. 

Primeiramente repetimos os testes da tarefa 2 neste ambiente, para garantir que o RTT fosse o mesmo. 

Usando o cliente e o servidor da tarefa 2, com um processo que envia e outro que recebe os dados, obtivemos um resultado de 7.519926 segundos para enviar o arquivo \/etc\/services. 

Repetimos o experimento usando os programas da tarefa 4. Com o servidor daemon e o cliente usando select obtivemos um tempo de 7.377163 segundos. 

Com isso verificamos que usando select o desempenho é tão bom quanto usando multiplos processos. Na verdade o resultado foi até um pouco melhor.


%%%%%%%%%%%%%%%%%%%%%%%%%%%%
\section{Conclusões}
%%%%%%%%%%%%%%%%%%%%%%%%%%%%

Notamos que o desempenho de um cliente usando select chega a ser mais rápido que um que usa multiplos processos. A pequena diferença de tempo pode ser devida ao servidor ser um daemon, e por isso estar desaclopado de terminais. 

Aprendemos que select() é uma solução viável para multiplexar comunicação de um servidor com vários clientes ou até mesmo para um simples cliente de echo.

%%%%%%%%%%%%%%%%%%%%%%%%%%%%
\section{Bibliografia}
%%%%%%%%%%%%%%%%%%%%%%%%%%%%

\bibliographystyle{plain}
\bibliography{bibliography}

%%%%%%%%%%%%%%%%%%%%%%%%%%%%
\section{Anexos}
%%%%%%%%%%%%%%%%%%%%%%%%%%%%

Os códigos dos programas seguem anexos.

\end{document}


